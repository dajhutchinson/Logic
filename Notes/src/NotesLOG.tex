\documentclass[11pt,a4paper]{article}

\usepackage[margin=1in, paperwidth=8.3in, paperheight=11.7in]{geometry}
\usepackage{amsfonts}
\usepackage{amsmath}
\usepackage{amssymb}
\usepackage{dsfont}
\usepackage{enumerate}
\usepackage{enumitem}
\usepackage{fancyhdr}
\usepackage{graphicx}
\usepackage{tikz}
\usepackage{changepage} 

\begin{document}

\pagestyle{fancy}
\setlength\parindent{0pt}
\allowdisplaybreaks

\renewcommand{\headrulewidth}{0pt}

% Cover page title
\title{Logic - Notes}
\author{Dom Hutchinson}
\date{\today}
\maketitle

% Header
\fancyhead[L]{Dom Hutchinson}
\fancyhead[C]{Logic - Notes}
\fancyhead[R]{\today}

% Counters
\newcounter{definition}[subsection]
\newcounter{example}[section]
\newcounter{notation}[section]
\newcounter{proposition}[section]
\newcounter{proof}[section]
\newcounter{remark}[section]
\newcounter{theorem}[section]

% commands
\newcommand{\dotprod}[0]{\boldsymbol{\cdot}}
\newcommand{\cosech}[0]{\mathrm{cosech}\ }
\newcommand{\cosec}[0]{\mathrm{cosec}\ }
\newcommand{\sech}[0]{\mathrm{sech}\ }
\newcommand{\prob}[0]{\mathbb{P}}
\newcommand{\nats}[0]{\mathbb{N}}
\newcommand{\cov}[0]{\mathrm{Cov}}
\newcommand{\var}[0]{\mathrm{Var}}
\newcommand{\expect}[0]{\mathbb{E}}
\newcommand{\reals}[0]{\mathbb{R}}
\newcommand{\integers}[0]{\mathbb{Z}}
\newcommand{\indicator}[0]{\mathds{1}}
\newcommand{\nb}[0]{\textit{N.B.} }
\newcommand{\ie}[0]{\textit{i.e.} }
\newcommand{\eg}[0]{\textit{e.g.} }
\newcommand{\X}[0]{\textbf{X}}
\newcommand{\x}[0]{\textbf{x}}
\newcommand{\iid}[0]{\overset{\text{iid}}{\sim}}
\newcommand{\proved}[0]{$\hfill\square$\\}

\newcommand{\definition}[1]{\stepcounter{definition} \textbf{Definition \arabic{section}.\arabic{definition}\ - }\textit{#1}\\}
\newcommand{\definitionn}[1]{\stepcounter{definition} \textbf{Definition \arabic{section}.\arabic{definition}\ - }\textit{#1}}
\newcommand{\proof}[1]{\stepcounter{proof} \textbf{Proof \arabic{section}.\arabic{proof}\ - }\textit{#1}\\}
\newcommand{\prooff}[1]{\stepcounter{proof} \textbf{Proof \arabic{section}.\arabic{proof}\ - }\textit{#1}}
\newcommand{\example}[1]{\stepcounter{example} \textbf{Example \arabic{section}.\arabic{example}\ - }\textit{#1}\\}
\newcommand{\examplee}[1]{\stepcounter{example} \textbf{Example \arabic{section}.\arabic{example}\ - }\textit{#1}}
\newcommand{\notation}[1]{\stepcounter{notation} \textbf{Notation \arabic{section}.\arabic{notation}\ - }\textit{#1}\\}
\newcommand{\notationn}[1]{\stepcounter{notation} \textbf{Notation \arabic{section}.\arabic{notation}\ - }\textit{#1}}
\newcommand{\proposition}[1]{\stepcounter{proposition} \textbf{Proposition \arabic{section}.\arabic{proposition}\ - }\textit{#1}\\}
\newcommand{\propositionn}[1]{\stepcounter{proposition} \textbf{Proposition \arabic{section}.\arabic{proposition}\ - }\textit{#1}}
\newcommand{\remark}[1]{\stepcounter{remark} \textbf{Remark \arabic{section}.\arabic{remark}\ - }\textit{#1}\\}
\newcommand{\remarkk}[1]{\stepcounter{remark} \textbf{Remark \arabic{section}.\arabic{remark}\ - }\textit{#1}}
\newcommand{\theorem}[1]{\stepcounter{theorem} \textbf{Theorem \arabic{section}.\arabic{theorem}\ - }\textit{#1}\\}
\newcommand{\theoremm}[1]{\stepcounter{theorem} \textbf{Theorem \arabic{section}.\arabic{theorem}\ - }\textit{#1}}

\tableofcontents

% Start of content
\newpage

\section{Introduction}

\subsection{Alphabets \& Strings}

\definition{Alphabet}
An \textit{Alphabet} is a set of symbols from which \textit{Strings} can be created.\\

\definition{String}
A \textit{String} over a set $\mathcal{A}$ is any sequence $\alpha:=\langle a_1,\dots,a_n\rangle$ where $a_1,\dots,a_n\in\mathcal{A}$.\\
\nb Here we say $\alpha$ has \textit{length} $n$ and $\alpha\in\mathcal{A}^n$.\\

\remark{Concatenating Strings}
Define \textit{Strings} $\alpha:=\langle a_1,\dots,a_n\rangle\in\mathcal{A}^n$ and $\beta:=\langle b_1,\dots,b_m\rangle\in\mathcal{A}^m$.\\
We define \textit{Concatenation} of $\alpha$ \& $\beta$ as
$\alpha\beta:=\langle a_1,\dots,a_n,b_1,\dots,b_m\rangle$
Note that
$$\alpha\beta\neq\langle\alpha,\beta\rangle=\langle\langle a_1,\dots,a_n\rangle,\langle b_1,\dots,b_m\rangle\rangle$$
\nb Sometimes the following notation is used $\alpha*\beta$.

\example{English Alphabet}
If we define an alphabet $\mathcal{A}:=\{`a`,\dots,`z`\}$ then $\langle`t`,`h`,`i`,`s`\rangle$ is a \textit{String} of $\mathcal{A}$.\\

\remark{Ambiguity when using multiple Alphabets}
Consider the \textit{Alphabets} $\mathcal{A}_1:=\{0,1,\dots,9\}$ \& $\mathcal{A}_2:=\nats$.\\
Then we are unsure which of the following definitions of $123$ is valid
$$\langle123\rangle,\ \langle12,3\rangle,\ \langle1,23\rangle, \langle1,2,3\rangle$$

\remark{$\mathcal{A}:=\{0,1\}$ is sufficient to describe any language - binary}

\remark{Describing Formal Languages}
When describing a \textit{Formal Language} we need to provide two things
\begin{enumerate}
	\item An \textit{Alphabet} which defines what symbols are allowed.
	\item A \textit{Grammar} which defines what combinations of symbols are allowed.
\end{enumerate}

\end{document}
