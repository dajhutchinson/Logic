\documentclass[11pt,a4paper]{article}

\usepackage[margin=1in, paperwidth=8.3in, paperheight=11.7in]{geometry}
\usepackage{amsfonts}
\usepackage{amsmath}
\usepackage{amssymb}
\usepackage{dsfont}
\usepackage{enumerate}
\usepackage{enumitem}
\usepackage{fancyhdr}
\usepackage{graphicx}
\usepackage{tikz}
\usepackage{changepage} 

\begin{document}

\pagestyle{fancy}
\setlength\parindent{0pt}
\allowdisplaybreaks

\renewcommand{\headrulewidth}{0pt}
\hyphenpenalty 10000
\exhyphenpenalty 10000

% Cover page title
\title{Logic - Problem Sheet 1}
\author{Dom Hutchinson}
\maketitle

% Header
\fancyhead[L]{Dom Hutchinson}
\fancyhead[C]{Logic - Problem Sheet 1}
\fancyhead[R]{\today}

% Counters
\newcounter{qpart}[section]

% commands
\newcommand{\dotprod}[0]{\boldsymbol{\cdot}}
\newcommand{\cosech}[0]{\mathrm{cosech}\ }
\newcommand{\cosec}[0]{\mathrm{cosec}\ }
\newcommand{\sech}[0]{\mathrm{sech}\ }
\newcommand{\prob}[0]{\mathbb{P}}
\newcommand{\nats}[0]{\mathbb{N}}
\newcommand{\cov}[0]{\mathrm{cov}}
\newcommand{\var}[0]{\mathrm{var}}
\newcommand{\expect}[0]{\mathbb{E}}
\newcommand{\reals}[0]{\mathbb{R}}
\newcommand{\integers}[0]{\mathbb{Z}}
\newcommand{\indicator}[0]{\mathds{1}}
\newcommand{\nb}[0]{\textit{N.B.} }
\newcommand{\ie}[0]{\textit{i.e.} }
\newcommand{\eg}[0]{\textit{e.g.} }
\newcommand{\iid}[0]{\overset{\text{iid}}{\sim} }
\newcommand{\x}[0]{\textbf{x} }
\newcommand{\X}[0]{\textbf{X} }
\newcommand{\LL}[0]{\mathcal{L}}

\newcommand{\qpart}[0]{\stepcounter{qpart} \textbf{Question \arabic{section} \alph{qpart})\\}}
\newcommand{\qpartnb}[0]{\stepcounter{qpart} \textbf{Question \arabic{section} \alph{qpart})} - }
\newcommand{\ans}[0]{ \textbf{Answer \arabic{section}\\}}
\newcommand{\apart}[0]{ \textbf{Answer \arabic{section} \alph{qpart})\\}}
\newcommand{\apartnb}[0]{ \textbf{Answer \arabic{section} \alph{qpart})} - }
\newcommand{\question}[0]{\stepcounter{section}\subsection*{Question - \arabic{section}.}}

\question
Prove
\begin{center}
If the sets $M_0,M_1,M_2,\dots$ are countable, then $\bigcup\limits_{n\in\nats}M_n$ is countable as well.
\end{center}

\ans

\question
Let $\LL=\{c,f,g\}$ be a first-order language, where $c$ is a constant symbol, $f$ is a binary function symbol and $g$ is a unary function symbol. Let $x\ \&\ y$ be variables.\\
\\
Prove that the following strings over $\mathcal{A_L}$ are $\LL$-terms and calculate their complexity.\\

\qpartnb $g(f(x,c))$.\\

\apart
$x,c$ are atomic $\LL$-terms with $cp(x)=cp(c)=0$.\\
Thus $f(x,c)$ is an $\LL$-term with $cp(f(x,c))=\max\{cp(x),cp(c)\}+1=\max\{0,0\}+1=1$.\\
Thus $g(f(x,c))$ is an $\LL$-term with $cp(g(f(x,c)))=\max\{cp(f(x,c)\}+1=1+1=2$.\\

\qpartnb $f(f(x,c),f(x,f(x,y)))$.\\

\apart
$x,y,c$ are atomic $\LL$-terms with $cp=0$.\\
Thus $f(x,y)$ is an $\LL$-term with $cp=\max\{cp(x),cp(y)\}+1=\max\{0,0\}+1=1$,\\
and $f(x,c)$ is an $\LL$-term with $cp=\max\{cp(x),cp(c)\}+1=0+1=1$.\\
Thus $f(x,f(x,y))$ is an $\LL$-term with $cp=\max\{cp(x),cp(f(x,y))\}=1+1=2$.\\
Thus $f(f(x,c),f(x,f(x,y)))$ is an $\LL$-term with
$$cp(f(f(x,c),f(x,f(x,y))))=\max\{cp(f(x,c)),cp(f(x,f(x,y)))\}=2+1=3$$

\question
Let $\LL=\{P,Q,c,f,g\}$ be a first-order language where $c,f\ \&\ g$ are as in $\mathtt{Question 2}$, $P$ is a unary predicate symbol and $Q$ is a binary predicate symbol.\\
\\
Prove that the following strings over $\mathcal{A_L}$ are $\LL$-formulae and calculate their complexity.\\

\qpartnb $\neg\forall x\neg\forall y(P(g(f(x,c)))\longrightarrow\equiv(y,y)$.\\

\apart
$x,y,c$ are $\LL$-Terms with $cp=0$.\\
Thus $f(x,c)$ \& $\equiv(y,y)$ are $\LL$-Terms with $cp=0$.\\
Thus $g(f(x,c))$ is an $\LL$-Term with $cp=0$.\\
Thus $P(g(f(x,c))$ is an $\LL$-Formula with $cp=0$.\\
Thus $\forall y(P(g(f(x,c)))$ is an $\LL$-Formula with $cp=0+1=1$.\\
Thus $\neg\forall y(P(g(f(x,c)))$ is an $\LL$-Formula with $cp=1+1=2$.\\
Thus $\forall x\neg\forall y(P(g(f(x,c)))$ is an $\LL$-Formula with $cp=2+1=3$.\\
Thus $\neg\forall x\neg\forall y(P(g(f(x,c)))$ is an $\LL$-Formula with $cp=3+1=4$.\\
Thus $\neg\forall x\neg\forall y(P(g(f(x,c)))\longrightarrow\equiv(y,y)$ is an $\LL$-Formula with $cp=\max\{4,0\}+1=5$.\\

\qpartnb $\big(\forall x\neg P\big(f(x,c)\big)\to Q\big(f(x,c),f(f(x,c),f(x,f(x,y)))\big)\big)$.\\

\apart
$x,y,c$ are $\LL$-Terms with $cp=0$.\\
Thus $f(x,c)\ \&\ f(x,y)$ are $\LL$-Terms with $cp=0$.\\
Thus $f(x,f(x,y))$ is an $\LL$-Term with $cp=0$.\\
Thus $f(f(x,c),f(x,f(x,y)))$ is an $\LL$-Term with $cp=0$.\\
Thus $P(f(x,c))$ \& $Q\big(f(x,c),f(f(x,c),f(x,f(x,y)))$ are $\LL$-Terms with $cp=0$.\\
Thus $\neg P(f(x,c))$ is an $\LL$-Formula with $cp=0+1=1$.\\
Thus $\forall x\neg P(f(x,c))$ is an $\LL$-Formula with $cp=1+1=2$.\\
Thus $\big(\forall x\neg P\big(f(x,c)\big)\to Q\big(f(x,c),f(f(x,c),f(x,f(x,y)))\big)\big)$ is an $\LL$-Formula with ${cp=\max\{2,0\}+1=3}$.

\question
Prove
\begin{center}
Every $\LL$-formula contains as many left parenthese as right parentheses.
\end{center}

\ans

\question
Let $x,y,z$ be variables and $\LL=\{f,P,Q,R\}$ where $f$ is a unary function symbol, $P$ is a binary predicate symbol, $Q$ is a unary predicate symbol and $R$ is a ternary predicate symbol.\\
For the following $\LL$-formulae, $\phi$, $\text{determine the corresponding set of variables that occur free in }\phi$.\\

\qpartnb $\forall x\exists y (P(x,z)\to\neg Q(y))\to\neg Q(y)$.\\

\apartnb
\[\begin{array}{rcl}
&&FV\big(\forall x\exists y (P(x,z)\to\neg Q(y))\to\neg Q(y)\big)\\
&=&FV((P(x,z)\to\neg Q(y))\to\neg Q(y))\backslash\{x,y\}\\
&=&[FV(P(x,z)\to\neg Q(y))\cup FV(\neg Q(y))]\backslash\{x,y\}\\
&=&[FV(P(x,z))\cup FV(\neg Q(y))\cup \{y\}]\backslash\{x,y\}\\
&=&[\{x,z\}\cup FV(Q(y))\cup \{y\}]\backslash\{x,y\}\\
&=&[\{x,z\}\cup\{y\}\cup \{y\}]\backslash\{x,y\}\\
&=&\{x,y,z\}\backslash\{x,y\}\\
&=&\{z\}
\end{array}\]

\qpartnb $\forall x\forall y(Q(c)\wedge Q(f(x)))\to\forall y\forall x(Q(y)\wedge R(x,x,y))$.\\

\apartnb
\[\begin{array}{rcl}
&&FV\big(\forall x\forall y(Q(c)\wedge Q(f(x)))\to\forall y\forall x(Q(y)\wedge R(x,x,y))\big)\\
&=&FV(\forall x\forall y(Q(c)\wedge Q(f(x))))\cup FV(\forall y\forall x(Q(y)\wedge R(x,x,y)))\\
&=&[FV(Q(c)\wedge Q(f(x)))\backslash\{x,y\}]\cup[FV(Q(y)\wedge R(x,x,y))\backslash\{x,y\}]\\
&=&[[FV(Q(c))\cup FV(Q(f(x)))]\backslash\{x,y\}]\cup[[FV(Q(y))\cup FV(R(x,x,y))]\backslash\{x,y\}]\\
&=&[[\emptyset\cup\{x\}]\backslash\{x,y\}]\cup[\{y\}\cup\{x,y\}]\backslash\{x,y\}]\\
&=&\emptyset\cup\emptyset\\
&=&\emptyset
\end{array}\]

\qpartnb $Q(z)\longleftrightarrow\exists z(P(x,y)\wedge R(c,x,y))$.\\

\apart
Consider
\[\begin{array}{rcl}
FV(\phi\longleftrightarrow\psi)&=&FV[(\phi\to\psi)\wedge(\psi\to\phi)]\\
&=&FV(\phi\to\psi)\cup FV(\psi\to\phi)\\
&=&FV(\phi)\cup FV(\psi)\cup FV(\psi)\cup FV(\phi)\\
&=&FV(\phi)\cup FV(\psi)
\end{array}\]
Thus
\[\begin{array}{rcl}
&&FV\big(Q(z)\longleftrightarrow\exists z(P(x,y)\wedge R(c,x,y))\big)\\
&=&FV(Q(z))\cup FV(\exists z(P(x,y)\wedge R(c,x,y)))\\
&=&\{z\}\cup[FV(P(x,y)\wedge R(c,x,y))\backslash\{z\}]\\
&=&\{z\}\cup[FV(P(x,y))\cup FV(R(c,x,y))\backslash\{z\}]\\
&=&\{z\}\cup\big[[\{x,y\}\cup\{x,y\}]\backslash\{z\}\big]\\
&=&\{z\}\cup\{x,y\}\\
&=&\{x,y,z\}
\end{array}\]

\qpartnb Which of these formulae are $\LL$-sentences?\\

\apartnb Only the formula from $\mathtt{(b)}$ is an $\LL$-sentence since it is the only one with no free variables.

\end{document}