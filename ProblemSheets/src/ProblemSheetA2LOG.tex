\documentclass[11pt,a4paper]{article}

\usepackage[margin=1in, paperwidth=8.3in, paperheight=11.7in]{geometry}
\usepackage{amsfonts}
\usepackage{amsmath}
\usepackage{amssymb}
\usepackage{dsfont}
\usepackage{enumerate}
\usepackage{enumitem}
\usepackage{fancyhdr}
\usepackage{graphicx}
\usepackage{nccmath}
\usepackage{tikz}
\usepackage{changepage} 

\begin{document}

\pagestyle{fancy}
\setlength\parindent{0pt}
\allowdisplaybreaks

\renewcommand{\headrulewidth}{0pt}
\hyphenpenalty 10000
\exhyphenpenalty 10000

% Cover page title
\title{Logic - Problem Sheet 2}
\author{Dom Hutchinson}
\maketitle

% Header
\fancyhead[L]{Dom Hutchinson}
\fancyhead[C]{Logic - Problem Sheet 2}
\fancyhead[R]{\today}

% Counters
\newcounter{qpart}[section]

% commands
\newcommand{\dotprod}[0]{\boldsymbol{\cdot}}
\newcommand{\cosech}[0]{\mathrm{cosech}\ }
\newcommand{\cosec}[0]{\mathrm{cosec}\ }
\newcommand{\sech}[0]{\mathrm{sech}\ }
\newcommand{\prob}[0]{\mathbb{P}}
\newcommand{\nats}[0]{\mathbb{N}}
\newcommand{\cov}[0]{\mathrm{cov}}
\newcommand{\var}[0]{\mathrm{var}}
\newcommand{\expect}[0]{\mathbb{E}}
\newcommand{\reals}[0]{\mathbb{R}}
\newcommand{\integers}[0]{\mathbb{Z}}
\newcommand{\indicator}[0]{\mathds{1}}
\newcommand{\nb}[0]{\textit{N.B.} }
\newcommand{\ie}[0]{\textit{i.e.} }
\newcommand{\eg}[0]{\textit{e.g.} }
\newcommand{\iid}[0]{\overset{\text{iid}}{\sim} }
\newcommand{\x}[0]{\textbf{x} }
\newcommand{\X}[0]{\textbf{X} }
\newcommand{\LL}[0]{\mathcal{L} }
\newcommand{\M}[0]{\mathfrak{M} }

\newcommand{\qpart}[0]{\stepcounter{qpart} \textbf{Question \arabic{section} \alph{qpart})\\}}
\newcommand{\qpartnb}[0]{\stepcounter{qpart} \textbf{Question \arabic{section} \alph{qpart})} - }
\newcommand{\ans}[0]{ \textbf{Answer \arabic{section}\\}}
\newcommand{\apart}[0]{ \textbf{Answer \arabic{section} \alph{qpart})\\}}
\newcommand{\apartnb}[0]{\textbf{Answer \arabic{section} \alph{qpart})} - }
\newcommand{\question}[0]{\stepcounter{section}\section*{Question - \arabic{section}.}}

\question
Let $\LL=\{P,R,f,g,c_0,c_1\}$ where $P$ is a unary predicate, $R$ is a binary prediate and $f\ \&\ g$ are binary function symbols. Let $\M$ be an $\LL$-Structure such that $|\M|=\reals,\ P^\M=\nats,\ R^\M$ is the usual greater-than $(>)$ relation on $R$, $f^\M$ is the usual addition on $\reals$, $g^\M$ is the usual multiplicaiton on $\reals$, $c_0^\M=0$ and $c_1^\M=1$. Finally, let $s$ be a variable assignment over $\M$ with $s(x)=5$ and $s(y)=3$ where $x$ and $y$ are distinct variables.\\

\stepcounter{qpart}
\qpartnb Determine the semantic value of the following $\LL$-term in $\M$ under $s$
$$f(g(x,y),g(x,c_1))$$

\apart
\[\begin{array}{rl}
&\bar{s}(f(g(x,y),g(x,c_1)))\\
\Longleftrightarrow&f^\M(\bar{s}(g(x,y)),\bar{s}(g(x,c_1)))\\
\Longleftrightarrow&f^\M(g^\M(\bar{s}(x),\bar{s}(y)),g^\M(\bar{s}(x),\bar{s}(c_1)))\\
\Longleftrightarrow&f^\M(\cdot^\M(s(x),s(y)),\cdot^\M(s(x),c_1^\M))\\
\Longleftrightarrow&+(\cdot(5,3),\cdot(5,1))\\
\Longleftrightarrow&+(15,5)\\
\Longleftrightarrow&20
\end{array}\]

\qpartnb Answer whether or not the $\LL$-formula below is satisfied in $\M$ under $s$
$$\forall x\forall y(R(x,c_0)\to\exists z(P(z)\wedge R(g(z,x),y)))$$

\apart
\[\begin{array}{rl}
&\M,s\vDash\forall x\forall y(R(x,c_0)\to\exists z(P(z)\wedge R(g(z,x),y)))\\
\Longleftrightarrow&\text{for all }d_0,d_1\in|\M|,\ \M,s\frac{d_0}{x}\frac{d_1}{y}\vDash(R(x,c_0)\to\exists z(P(z)\wedge R(g(z,x),y)))\\
\Longleftrightarrow&\text{for all }d_0,d_1\in\reals\text{ if }\M,s\frac{d_0}{x}\frac{d_1}{y}\vDash R(x,c_0)\text{ then }\M,s\frac{d_0}{x}\frac{d_1}{y}\vDash\exists z(P(z)\wedge R(g(z,x),y))\\
\Longleftrightarrow&\text{for all }d_0,d_1\in\reals\text{ if }d_0>0\text{ then exists }d_2\in|\M|\text{ st }\M,s\frac{d_0}{x}\frac{d_1}{y}\frac{d_2}{z}\vDash (P(z)\wedge R(g(z,x),y))\\
\Longleftrightarrow&\text{for all }d_0,d_1\in\reals\text{ if }d_0>0\text{ then exists }d_2\in\reals\text{ st }\M,s\frac{d_0}{x}\frac{d_1}{y}\frac{d_2}{z}\vDash P(z)\text{ and }\M,s\frac{d_0}{x}\frac{d_1}{y}\frac{d_2}{z}\vDash R(g(z,x),y)\\
\Longleftrightarrow&\text{for all }d_0,d_1\in\reals\text{ if }d_0>0\text{ then exists }d_2\in\reals\text{ st }d_2\in\nats\text{ and }\M,s\frac{d_0}{x}\frac{d_1}{y}\frac{d_2}{z}\vDash g(z,x)>y\\
\Longleftrightarrow&\text{for all }d_0,d_1\in\reals\text{ if }d_0>0\text{ then exists }d_2\in\nats\text{ st }d_2\cdot d_0>d_1\\
\Longleftrightarrow&\mathtt{true}
\end{array}\]

\question
Let $\LL=\{P,f\}$ where $P$ is a unary predicate and $f$ is a binary function symbol.\\
For each of the following $\LL$-formulae find an $\LL$-structure $\M_0$ and a variable assignemnt $s_0$ over $\M_0$ in which the formula is true, and find an $\LL$-structure $\M_1$ and a variable assignment $s_1$ over $\M_1$ in which the formula is false:\\

\stepcounter{qpart}
\qpartnb $\exists v_2\forall v_1 f(v_2,v_1)\equiv v_2$.\\

\apart
$\M_0=(\reals,\cdot),\ \bar{s}_0(v_2)=0$.\\
$\M_1=(\reals,\cdot),\ \bar{s}_1(v_2)=1$.\\

\qpartnb $\exists v_2(P(v_2)\wedge\forall v_1 P(f(v_2,v_1)))$.\\

\apart
$\M_0=(\reals,\equiv0,\cdot),\ \bar{s}_0(v_2)=0$.\\
$\M_1=(\reals,\equiv0,\cdot),\ \bar{s}_1(v_2)=1$.

\setcounter{section}{4}
\question
Show the following\\

\qpartnb $\vDash (\phi\to\forall x\phi)$ if $x\not\in FV(\phi)$.\\

\apart
We have that
$$\vDash(\phi\to\forall x\phi)\text{ if }x\not\in FV(\phi)\Longleftrightarrow\text{if }\vDash\phi\text{ then }\vDash\forall x\phi\text{ if }x\not\in FV(\phi)$$
Whether $\phi$ is satisfied depends on its free variables.\\
Since $x\not\in FV(\phi)$, whether $\phi$ is satisified is independent of the value of $x$.\\
Under the condition that $\phi$ is logically valid, it holds that $\forall x\phi$ is logically valid for $x\not\in FV(\phi)$.\\
Thus the statement holds.\\

\qpartnb $\vDash \forall x(\phi\to\psi)\to(\forall x\phi\to\forall x\psi)$.\\

\apart
\[\begin{array}{rl}
&\vDash \forall x(\phi\to\psi)\\
\Longleftrightarrow&\text{for all }x\in|\M|\vDash(\phi\to\psi)\\
\Longleftrightarrow&\text{for all }x\in|\M|\text{ if }\vDash\phi\text{ then}\vDash\psi\\
\Longleftrightarrow&\text{if }\text{for all }x\in|\M|\vDash\phi\text{ then for all }x\in|\M|\vDash\psi\\
\Longleftrightarrow&\vDash(\forall x\phi\to\forall x\psi)\\
\implies&\vDash \forall x(\phi\to\psi)\to\ \vDash(\forall x\phi\to\forall x\psi)\\
\Longleftrightarrow&\vDash \forall x(\phi\to\psi)\to(\forall x\phi\to\forall x\psi)
\end{array}\]

\qpartnb if $\Gamma\vDash\phi$ and $x\not\in FV(\Gamma):=\bigcup\limits_{\chi\in\Gamma}FV(\chi)$ then $\Gamma\vDash\forall x\phi$ where $\Gamma\subset\text{Fml}_\LL$.\\


\apart
%TODO
Assume that $\Gamma\vDash\phi$.\\
Then, for all models $\M,s$ of $\Gamma$, $\M,s\vDash\phi$.\\
Thus, for all models $\M,s$ where $\M,s\vDash\gamma$ for all $\gamma\in\Gamma$, $\M,s\vDash\phi$.\\
Thus, if $\M,s\vDash\gamma$ for all $\gamma\in\Gamma$ then $\M,s\vDash\phi$.\\
Let $x\not\in FV(\Gamma)$.\\
Then, if $\M,s\vDash\gamma$ then $\M,s\vDash\forall x\gamma$ by \textbf{Question 5 a)}.\\
Thus, if $\M,s\vDash\gamma$ for all $\gamma\in\Gamma$ then $\M,s\vDash(\phi\wedge\forall x\gamma)$ for all $\gamma\in\Gamma$.\\
Thus, if $\M,s\vDash\gamma$ for all $\gamma\in\Gamma$ then for all $x\not\in FV(\Gamma)$, $\M,s\vDash(\phi\wedge\gamma)$ for all $\gamma\in\Gamma$.\\
Thus, if $\M,s\vDash\gamma$ for all $\gamma\in\Gamma$ then for all $x\not\in FV(\Gamma)$, $\M,s\vDash(\phi)$.\\
Thus, if $\M,s\vDash\gamma$ for all $\gamma\in\Gamma$ then $\M,s\vDash(\forall x\phi)$ with $x\not\in FV(\Gamma)$.

\end{document}